\section*{\centering Reproducibility Summary}

\subsection*{Scope of Reproducibility}
Based on the intuition that attention in neural networks is what the model focuses on, attention is now being used as an explanation for a models' prediction (see \citet{galassi2020attention} for a survey). \citet{pruthi-etal-2020-learning} challenge the usage of attention-based explanation through a series of experiments using classification and sequence-to-sequence (seq2seq) models. They examine the model's use of impermissible tokens, which are user-defined tokens that can introduce bias e.g. gendered pronouns. Across multiple datasets, the authors show that with the impermissible tokens removed the model accuracy drops, implying their usage in prediction. And then by penalising attention paid to the impermissible tokens but keeping them in, they train models that retain full accuracy hence must be using the impermissible tokens, but that does not show attention being paid to the impermissible tokens. As the paper's claims have such significant implications for the use of attention-based explanations, we seek to reproduce their results.

\subsection*{Methodology}
Using the authors' code, for classifiers we attempt to reproduce their embedding, BiLSTM, and BERT results across the occupation prediction, gender identify, and SST + wiki datasets. Further, we reimplemented BERT using HuggingFace's transformer library \citep{huggingface} with restricted self-attention (information cannot flow between permissible and impermissible tokens). For seq2seq we used the authors' code to reproduce results across Bigram Flip, Sequence Copy, Sequence Reverse, and English-German (En-De) machine translation datasets. We performed refactoring on the authors' code aiming toward a more uniformly usable code style as well as porting across to PyTorch Lightning. All experiments were run in approximately 130 GPU hours on a computing cluster with nodes containing Titan RTX GPUs.

\subsection*{Results}
We reproduced the authors' results across all models and all available datasets, confirming their findings that attention-based explanations can be manipulated and that models can learn to deceive. We also replicated their BERT results using our reimplemented model. There was only one result not as strongly ($>$ 1 S.D.) in their experimental direction.

\subsection*{What Was Easy}
The authors' methods were largely well described and easy to follow, and we could quickly produce the first results as their code worked straight away with minor adjustments. They were also extremely responsive and helpful via email.

\subsection*{What Was Difficult}
Re-implementing the BERT-based classification model to perform replicability, with further specification details on model architecture, penalty mechanism, and training procedure needed. Also, porting code across to PyTorch Lightning.

\subsection*{Communication With Original Authors}
There was a continuous email chain with the authors for several weeks during the reproducibility work. They made additional code and datasets available per our requests, along with providing detailed responses and clarifications to our emailed questions. They encouraged the work and we wish to thank them for their time and support.

\newpage

% The following section formatting is \textbf{optional}, you can also define sections as you deem fit.
% Focus on what future researchers or practitioners would find useful for reproducing or building upon the paper you choose.

\section{Introduction}
% A few sentences placing the work in high-level context. Limit it to a few paragraphs at most; your report is on reproducing a piece of work, you don’t have to motivate that work.
Attention is a mechanism to automatically learn the relevance of different elements of the input to a model, rather than relying on manual feature engineering, allowing computational learning to focus on important elements \citep{galassi2020attention}.
Originally introduced to natural language processing (NLP) for neural machine translation \citep{bahdanau2014neural} its usage has since expanded. \citet{vaswani2017attention} termed it "all you need", having removed recurrence and convolutions and relied on attention in their Transformer architecture. Transformers are now in wide use, with \citet{devlin2018bert}'s Bidirectional Encoder Representations from Transformers (BERT) a commonly used model in NLP.

Because neural networks are subsymbolic with knowledge stored numerically, it is challenging to understand their inner workings \citep{galassi2020attention}. With interpretability a growing concern in NLP, there is a body of work on attention-based explanations of neural architectures using visualisation of attention weights \citep{serrano2019attention}. However, there is a rich and ongoing debate about whether attention is an explanation or not \citep{jain2019attention,wiegreffe2019attention}. Acknowledging the debate, \citet{pruthi-etal-2020-learning} whose work we seek to reproduce, examine whether models can learn to deceive, by adding a penalty to the loss function that punishes the model when attention is paid to impermissible tokens. These tokens are user-defined and may refer inter alia to terms for protected traits such as gender (the pronouns \textit{she}, \textit{her} etc.), sexual orientation, or race. Their research indicates that the impermissible tokens are still being used by the model as there is no accuracy drop seen, while there is one when these tokens are instead fully removed. Thus, the model is both able to use the impermissible tokens in learning and inference, but not pay attention to them. Hence bringing into question the validity of using attention in the explanation of a model's decision.

\section{Scope of Reproducibility}
\label{sec:claims}

% Introduce the specific setting or problem addressed in this work, and list the main claims from the original paper. Think of this as writing out the main contributions of the original paper. Each claim should be relatively concise; some papers may not clearly list their claims, and one must formulate them in terms of the presented experiments. (For those familiar, these claims are roughly the scientific hypotheses evaluated in the original work.)

% A claim should be something that can be supported or rejected by your data. An example is, ``Finetuning pretrained BERT on dataset X will have higher accuracy than an LSTM trained with GloVe embeddings.''
% This is concise, and is something that can be supported by experiments.
% An example of a claim that is too vague, which can't be supported by experiments, is ``Contextual embedding models have shown strong performance on a number of tasks. We will run experiments evaluating two types of contextual embedding models on datasets X, Y, and Z."

% This section roughly tells a reader what to expect in the rest of the report. Clearly itemise the claims you are testing:
% \begin{itemise}
%     \item Claim 1
%     \item Claim 2
%     \item Claim 3
% \end{itemise}

% Each experiment in Section~\ref{sec:results} will support (at least) one of these claims, so a reader of your report should be able to separately understand the \emph{claims} and the \emph{evidence} that supports them.

%\jdcomment{To organisers: I asked my students to connect the main claims and the experiments that supported them. For example, in this list above they could have ``Claim 1, which is supported by Experiment 1 in Figure 1.'' The benefit was that this caused the students to think about what their experiments were showing (as opposed to blindly rerunning each experiment and not considering how it fit into the overall story), but honestly it seemed hard for the students to understand what I was asking for.}

The core finding of the paper is that attention-based explanations of models can be deceptive by, for instance, hiding the model’s use of gendered pronouns at inference from an auditor. Specifically, the authors show that attention weights can be manipulated during training by penalising the allocation of attention to impermissible tokens, without this affecting model performance. Any resulting attention-based explanation might suggest that the model did not rely on impermissible tokens to make its predictions, when in reality the model still uses these but not through the attention mechanism, thereby making the model ``deceptive''. The key findings can be decomposed into the following claims which we are testing in our research:

\begin{enumerate}
    \item The attention mass on impermissible tokens can be reduced without significantly affecting the classification accuracy of Embeddings + Attention, BiLSTM + Attention, and BERT + Attention models across several tasks (see Tables \ref{tab:classification-results-1} and \ref{tab:classification-results-2}).
    \item The attention mass on impermissible tokens can be reduced without significantly affecting the seq2seq performance on translation as measured by BLEU (see Table \ref{tab:seq-to-seq translation results}).
    \item The attention mass on impermissible tokens can be reduced without significantly affecting the seq2seq accuracy on synthetic data tasks (see Tables \ref{tab:authors-seq-to-seq-results-1} and  \ref{tab:authors-seq-to-seq-results-2}).
\end{enumerate}

We chose not to perform the human study, testing whether visualised attention weights could deceive NLP/ML trained and Transformer knowledgeable participants, as we deemed that the small sample size does not add value to the results.

\section{Methodology}
\label{sec:methodology}
% Explain your approach - did you use the authors' code, or did you aim to re-implement the approach from the description in the paper? Summarise the resources (code, documentation, GPUs) that you used.
Initially, we attempted to reproduce the findings from the provided repository without contacting the authors. The authors use three classification models: embeddings with attention; BiLSTM with attention; and a BERT-based model with attention. For the seq2seq tasks, the model is an encoder-decoder architecture. Code for all models (except BERT) was available in the authors' repository, and aside from minor dependency issues in the environment file, we were able to successfully run experiments to reproduce the results. We re-implemented the BERT-based model to replicate their results, and after the BERT-based code was added to the repository, we also reproduced their results. Lastly, we refactored the existing code-base and ported the PyTorch-based code to PyTorch Lightning.

\subsection{Attention Manipulation}
To explicitly optimise the models to learn deceptive attention weights, the authors introduce an auxiliary loss component that penalises the model for attending to impermissible tokens. Impermissible tokens are user-defined from a corpus and are the set of words $\mathcal{I}$ that a model should not use during training or inference, as they might introduce bias or other ethical issues. An example is the use of gendered pronouns "\textit{her, she, Ms.}" which might lead a model to discriminate against a specific gender. The remaining words in the corpus are deemed permissible, and thus constitute the complement set $\mathcal{I}^c$. Assuming an input sequence $S = w_1, w_2, ..., w_n$ of $n$ tokens, the authors proceed to define a binary attention mask vector $m$ of length $|S|$, with each element denoting the occurrence of an impermissible token:
\[\boldsymbol{m}_i=
\begin{cases}
      1, & \text{if }w_i \in \mathcal{I} \\
      0 & \text{otherwise}
   \end{cases}
\]
Furthermore, we assume an attention vector $\boldsymbol{\alpha} \in [0,1]^n$ which denotes the allocated attention for each token in the input sequence. From this, the authors construct the additive task-agnostic penalty term $R$, such that $L' = L + R$, where $R$ captures the extent to which a model's attention layer is penalised for allocating attention to impermissible tokens:
$$R=-\lambda \log(1-\boldsymbol{\alpha}^T\boldsymbol{m})$$
Here, the $\boldsymbol{\alpha}^T\boldsymbol{m}$ term denotes the attention allocated to impermissible words. Taking the negative log of the complement of this term then allows us to minimise this quantity via standard gradient descent. Furthermore, $\lambda$ is a coefficient that is used to control the extent to which impermissible attention allocation is penalised. Following the authors' methods, we consider values for $\lambda = \{0, 0.1, 1.0\}$. For models featuring multi-head attention (such as BERT), the authors use two different penalty variants. Namely, $R_{mean}$ optimises the mean of the penalty over the set of all heads $\mathcal{H}$, while $R_{max}$ instead only considers the head which allocates the most attention to impermissible tokens:
$$R_{mean} = -\frac{\lambda}{|\mathcal{H}|}\sum_{h\in \mathcal{H}} \log(1-\boldsymbol{\alpha}_{h}^{T}\boldsymbol{m}), \ \ \ \ \ \ \ \ \ R_{max}=-\lambda \cdot \min_{h \in \mathcal{H}} \log(1-\boldsymbol{\alpha}_{h}^{T}\boldsymbol{m})$$
Note that for the multi-head penalties, only the heads from the model's last layer are considered, thus in BERT $\alpha$ is defined as the attention paid by the [CLS] token to the other tokens.

\subsection{Model Descriptions}
\textbf{Embedding + Attention} This model serves as a baseline to the other models used for classification. It contains between 2.7M and 6.5M parameters depending on the dataset and consists of a dot-product attention mechanism applied on word embeddings. The resulting attention vector is then passed into a linear classifier followed by a softmax activation. The size of the embedding in the original paper was 128. We use cross-entropy loss and the Adam optimizer with a learning rate of 0.001 and no weight decay. \cite{pruthi-etal-2020-learning} argue that the accuracy of the Embedding and BiLSTM models could have been greatly impacted by the lambda parameter because those models might be under-parameterised for the SST-Wiki dataset. To study this we also train models with embedding sizes of 256 and 512.

\textbf{BiLSTM + Attention} The model still uses word embeddings and consists of a dot-product attention mechanism applied on the output of a bidirectional LSTM \citep{bilstm} The resulting attention vector is then passed into a linear classifier followed by a softmax activation. Again, the size of the embedding is 128, but we also train models with embedding sizes of 256 and 512. Its number of parameters is between 2.8M and 6.6M parameters depending on the size of the vocabulary of the given dataset. The BiLSTM was trained using the same hyper-parameters as the Embedding model above, with the dimension of the hidden state being 64.

\textbf{Transformer Models} For the transformer-based architecture, we use BERT \citep{devlin2018bert}. Specifically, we use a pre-trained instance of \texttt{BERT-base-uncased}, which consists of 12 transformer blocks (each with 12 heads) amounting to 109M trainable parameters. We trained each model for 10 epochs, with a batch size of 32. All models were optimised using Adam, with a learning rate of $5e-5$. Furthermore, we applied dropout with $p = 0.3$ to improve model generalisation. A sequence classification layer is added on top of the architecture, to adapt the model for sentence classification. Following the authors' methods, we apply a self-attention mask $M$ to the self-attention probabilities via element-wise multiplication in the models' forward pass, to avoid information flowing between the sets of impermissible tokens $\mathcal{I}$ and permissible tokens $\mathcal{I}^c$. Specifically, $M$ is a binary matrix of size $n \times n$, where $n$ denotes the sequence length, and where elements $M_{p,q}$ are 1 if both tokens $w_p$ and $w_q$ belong to the same set (either $\mathcal{I}$ or $\mathcal{I}^c$), and 0 otherwise. Moreover, the first column of $M$, which denotes the extent to which all other tokens attend to the \texttt{[CLS]} token, is zero also, as this further restricts the flow of information between tokens from $\mathcal{I}$ and $\mathcal{I}^c$ via the \texttt{[CLS]} token.

\textbf{Seq2seq} \cite{pruthi-etal-2020-learning} provide a bidirectional and unidirectional Gated Recurrent Unit (GRU) with dot-product attention respectively for their encoder-decoder model tackling seq2seq tasks. The input is passed through the encoder and decoder, where the final hidden state from the bidirectional GRU fed through a linear layer is the initial hidden state to the decoder. The embedding size was 256 and hidden size 512 for both encoder and decoder. We also use a teacher forcing ratio of 0.5 as well as the top-1 greedy strategy for decoding output sequences. For baseline experiments, this model is also trained with no attention and uniform attention overall source tokens. It overall contains 8.7M parameters for the synthetic datasets and 48.55M parameters for the Multi30K dataset, which are both described in Section \ref{sec:datasets}.

\subsection{Datasets}
\label{sec:datasets}
% For each dataset include 1) relevant statistics such as the number of examples and label distributions, 2) details of train / dev / test splits, 3) an explanation of any preprocessing done, and 4) a link to download the data (if available).

The original work features 8 tasks with associated datasets. For the classification models, these are Occupational Prediction, Gender Identity, SST + Wiki, and Reference Letters. For the seq2seq experiments, three synthetic datasets were used; Bigram Flip, Sequence Copy, and Sequence Reverse tasks. Additionally, the Multi30K dataset \citep{elliott2016multi30k} was used for English to German machine translation (MT). All of the datasets were available in the authors' repository except for Reference Letters, with the authors citing privacy concerns. Consequently, we were not able to reproduce this experiment. For Occupation Prediction the authors state that downsampling by a factor of 10 was done for minority classes. As it was not clear from the data provided in the repository if downsampling had already been applied, the authors confirmed via email that this was the case. No further pre-processing was required, besides that already present in the authors' code. Details of the datasets used in our experiments are provided in Table \ref{tab:dataset-table}.

\begin{table}
  \caption{Details of datasets.}
  \label{tab:dataset-table}
  \centering
  	\setlength\tabcolsep{5pt}
\begin{tabular}{llccccc}
\toprule
 Task Type               &    Dataset                   & Examples & Train & Val  & Test & Label Dist.\\ \midrule
Classification & Occupation Pred. &     25185       & 17629         & 2519      & 5037       & 68-32 \\
               & Gender Identity       &    11271       & 9017          & 1127      & 1127       & 50-50   \\
               & SST + Wiki            &    9613        & 6920          & 872       & 1821       & 48-52   \\ \midrule
Seq-to-Seq     & Bigram Flip           &    300000      &   100000      &  100000   & 100000  &   -       \\
               & Sequence Copy         &    300000      &   100000      &   100000  & 100000  &   -     \\
               & Sequence Reverse      &    300000      &   100000      &  100000   & 100000  &    -    \\
               & En-De Translation     &    31016       &    29001      &  1015     & 1000    &    -    \\
               \bottomrule
  \end{tabular}
\end{table}

\subsection{Hyperparameters}
% Describe how the hyperparameter values were set. If there was a hyperparameter search done, be sure to include the range of hyperparameters searched over, the method used to search (e.g. manual search, random search, Bayesian optimisation, etc.), and the best hyperparameters found. Include the number of total experiments (e.g. hyperparameter trials). You can also include all results from that search (not just the best-found results).

Except for the $\lambda$ coefficient (values 0.0, 0.1, and 1.0) as used in the computation of the regularising component $R$, the original work did not provide details regarding hyperparameters and/or tuning thereof. Upon contacting the authors, we learned that no hyperparameter tuning was performed, as the experimental findings could be achieved with conventional parameters. Therefore, in reproducing their experiments we have used the same standard configurations as the authors. %We trained the simple attention model using an embedding size of 128, the cross entropy loss and the Adam optimizer with a learning rate of 0.001 and no weight decay. The BiLSTM was trained using the same hyper-parameters with the dimension of the hidden state being 64.%For BERT, we trained each model for 10 epochs, with a batch size of 32. All models were optimised using Adam, with a learning rate of $5e-5$. Furthermore, we applied dropout with $p = 0.3$ to improve model generalisation.


% (to-do: provide overview of hyperparameters per model)

\subsection{Experimental Setup and Code}
% Include a description of how the experiments were set up that's clear enough a reader could replicate the setup.
% Include a description of the specific measure used to evaluate the experiments (e.g. accuracy, precision@K, BLEU score, etc.).
% Provide a link to your code.
The code used to reproduce the experiments can be found in this Github repository~\footnote{https://github.com/MatPrst/FACT}. The authors' negative baseline, the first row of each model in Table 3 of the original paper, was produced by removing the impermissible tokens (anonymising or deleting). They show that the performance of the model dropped. This drop was in comparison to the true baseline in row two, which provides the models' performance when impermissible words are freely used with no manipulation penalty applied i.e. $\lambda=0.0$. The third and fourth rows provide results for adjusting the penalty coefficient to 0.1 and 1.0 respectively. To reproduce the experiments by anonymising or removing the impermissible tokens, we had to look deeper into their script. For the Occupation Prediction and Gender Identity datasets, the authors provided an anonymisation functionality that transformed all pronouns to gender-neutral ones. However, for the SST + Wiki dataset, we had to implement the functionality to remove the SST sentence because it was not present in the scripts. Similarly, we added the functionality to the training script for the provided BERT implementation.

% The classification experimental results in Table 3 of the original paper provide the averaged value over five runs (with five different seeds). The individual results for each run were not available in the repository.
The repository contained a bash script to run the experiments with Embedding + Attention and BiLSTM + Attention without removing or anonymising the impermissible tokens. We recreated the classification experiments using the seed values from this script. The training outputs of those experiments were not as presented in the README of the authors' repository and did not contain a clear attention mass value. However, the authors clarified that the ``attention ratio'' measure was used in the paper. From those experimental runs, we could determine the average and standard deviation of the five runs.

For the classification models, we chose to largely follow the authors' experimental setup: we run experiments with the same set of values for the loss coefficient $\lambda \in {0, 0.1, 1.0}$. Furthermore, all models are trained for 10 epochs and are evaluated on the development set after each epoch. Here, we measure two metrics; the validation accuracy, and the average attention mass over all examples. The model with $\lambda = 0$ serves as the baseline for the 'adversarial' models with $\lambda = 0.1, 1.0$, i.e. the models that are explicitly optimised to learn deceptive attention maps. For the BERT replication, when evaluating models on the test set, we follow the authors' heuristic, i.e. we select the checkpoint which is within 2\% of the baseline test accuracy, and which has the greatest reduction in attention mass on the validation set.

As per the authors' experimental setup we considered the four sequence-to-sequence tasks:  Bigram Flipping, Sequence Copying, and Sequence Reversal are synthetic tasks that work with input-output-mappings with the respective gold alignments considered as impermissible tokens. The models are trained on 100K random input sequences with length 32 from a vocabulary of 1000 tokens and validated and tested on 100K unseen random sequences. Machine translation from German to English acts as the fourth task for which gold alignments are not available. Thus, the Fast Align toolkit \citep{dyer-etal-2013-simple} was used by the authors to align target and source words. In this task, the aligned words are used as impermissible tokens.

The seq2seq experimental results in Table 4 of the original paper provide the averaged value over five runs. The different runs and their results were not available in the repository, however, after emailing the authors we were provided with the results for each of the five experimental runs in each condition, along with the seeds used. This allowed us to recreate the experiment using the same seeds, and to determine along with the average, whether the standard deviation between our results also matched.

Pertaining to the English to German translations, the BLEU score was used, however, this was not available in the repository. After contacting the authors, we were provided with a link to the BLEU library they had used which is called ``compare-mt'' \citep{neubig-etal-2019-compare}. We had meanwhile used the NLTK implementation \citep{BirdKleinLoper09}, presuming it to be the most likely used. Therefore, for translation, we have used two different BLEU implementations: compare-mt for reproduction and NLTK for replication.


\subsection{Computational Requirements}
\label{sec:infrastructure}
% Include a description of the hardware used, such as the GPU or CPU the experiments were run on.
% For each model, include a measure of the average runtime (e.g. average time to predict labels for a given validation set with a particular batch size).
% For each experiment, include the total computational requirements (e.g. the total GPU hours spent).
% (Note: you'll likely have to record this as you run your experiments, so it's better to think about it ahead of time). Generally, consider the perspective of a reader who wants to use the approach described in the paper --- list what they would find useful.

All experiments were run on the LISA computing cluster provided by SURFsara, which is available to University of Amsterdam Master students. The nodes used contained 4 x Titan RTX GPUs. A breakdown of the computation is provided in Table \ref{tab:computation-table}.

\begin{table}
  \caption{Breakdown of approximate computational requirements for running experiments per task, for a single seed.}
  \label{tab:computation-table}
  \centering
   \setlength\tabcolsep{5pt}
  \begin{tabular}{lc|cccc}
    \toprule
    \multicolumn{2}{c}{Classification} & \multicolumn{4}{c}{GPU Hours} \\
    \cmidrule(r){1-6}
    Model     & Batch size       & Occupation Pred.  & Gender Identity & SST + Wiki \\
    \midrule
    Embedding & 1     & 0.62  & 0.56 & 1.1 \\
    BiLSTM     & 1     & 0.94  & 1 & 1.3 \\
    BERT     & 32     & 3.1  & 1.5 & 1.2 \\
    BERT(HF)     & 32    & 4.8  & 1.9 & 2.5 \\
    \toprule
    \multicolumn{2}{c}{Seq-to-Seq} & \multicolumn{4}{c}{GPU Hours} \\
    \cmidrule(r){1-6}
    Model     & Batch size       & Bigram Flip     & Seq. Copy     & Seq. Reverse     & En-De \\
    \midrule
    Enc-Dec & 128  & 0.42  & 0.36 & 0.34 & 0.15 \\
    \bottomrule
  \end{tabular}
\end{table}

\section{Reproduction and Replication Results}
\label{sec:results}
% Start with a high-level overview of your results. Do your results support the main claims of the original paper? Keep this section as factual and precise as possible, reserve your judgement and discussion points for the next "Discussion" section.

\begin{table}
\centering
\caption{Classification results from Table 3 in \citet{pruthi-etal-2020-learning} for datasets Occupation Prediction and Gender Identity with cell scheme \textit{author | reproduced} for all models except BERT(HgFc) which follows cell scheme \textit{author | replicated}. Our values are means over 5 different seeds.}
\label{tab:classification-results-1}
\begin{tabular}{lccccccc} 
\toprule
Model      & $\lambda$ & I & \multicolumn{2}{c}{Occupation Prediction}                  &  & \multicolumn{2}{c}{Gender Identity}                    \\ 
\cline{4-5}\cline{7-8}
           &           &   & Accuracy                    & Attention Mass                        &  & Accuracy                    & Attention Mass                         \\ 
\midrule
Embedding  & 0.0       & X & $93.8\pmb{\,|\,}93.4$   & -                           &  & $66.8 \pmb{\,|\,} 71.0$ & -                            \\
Embedding  & 0.0       & \checkmark  & $96.3 \pmb{\,|\,} 96.5$ & $51.4 \pmb{\,|\,} 56.4$     &  & $100 \pmb{\,|\,} 100$   & $99.2 \pmb{\,|\,}90.1$       \\
Embedding  & 0.1       & \checkmark  & $96.2\pmb{\,|\,}96.3$   & $4.6\pmb{\,|\,}5.70$        &  & $99.4\pmb{\,|\,}99.9$   & $3.4\pmb{\,|\,}8.8$          \\
Embedding  & 1.0       & \checkmark  & $96.2\pmb{\,|\,}96.1$   & $1.3\pmb{\,|\,}1.50$        &  & $99.2\pmb{\,|\,}99.5$   & $0.8\pmb{\,|\,}4.6$          \\ 
\midrule
BiLSTM     & 0.0       & X & $93.3\pmb{\,|\,}93.6$   & -                           &  & $63.3\pmb{\,|\,}71.1$   & -                            \\
BiLSTM     & 0.0       & \checkmark  & $96.4\pmb{\,|\,}96.7$   & $50.3\pmb{\,|\,}44.1$       &  & $100\pmb{\,|\,}100$     & $96.8\pmb{\,|\,}95.5$        \\
BiLSTM     & 0.1       & \checkmark  & $96.4\pmb{\,|\,}96.6$   & $0.08\pmb{\,|\,}3.70$       &  & $100\pmb{\,|\,}100$     & $<10^{-6}\pmb{\,|\,}0.07$    \\
BiLSTM     & 1.0       & \checkmark  & $96.7\pmb{\,|\,}96.5$   & $<10^{-2}\pmb{\,|\,}0.015$  &  & $100\pmb{\,|\,}100$     & $<10^{-6}\pmb{\,|\,}0.0047$  \\ 
\midrule
BERT       & 0.0       & X & $95.0\pmb{\,|\,}95.8$   & -                           &  & $72.8\pmb{\,|\,}82.3$   & -                            \\
BERT(mean) & 0.0       & \checkmark  & $97.2\pmb{\,|\,}97.1$   & $13.9\pmb{\,|\,}9.10$       &  & $100\pmb{\,|\,}99.9$    & $80.8\pmb{\,|\,}55.1$        \\
BERT(mean) & 0.1       & \checkmark  & $97.2\pmb{\,|\,}97.3$   & $0.001\pmb{\,|\,}0.007$     &  & $99.9\pmb{\,|\,}99.9$   & $<10^{-3}\pmb{\,|\,}0.004$   \\
BERT(mean) & 1.0       & \checkmark  & $97.2\pmb{\,|\,}97.3$   & $<10^{-3}\pmb{\,|\,}0.0007$ &  & $99.9\pmb{\,|\,}99.9$   & $<10^{-3}\pmb{\,|\,}0.0003$  \\ 
\midrule
BERT       & 0.0       & X & $95.0\pmb{\,|\,}95.8$   & -                           &  & $72.8\pmb{\,|\,}82.3$   & -                            \\
BERT(max)  & 0.0       & \checkmark  & $97.2\pmb{\,|\,}97.1$   & $99.7\pmb{\,|\,}65.5$       &  & $100\pmb{\,|\,}99.9$    & $99.7\pmb{\,|\,}99.8$        \\
BERT(max)  & 0.1       & \checkmark  & $97.1\pmb{\,|\,}97.1$   & $<10^{-3}\pmb{\,|\,}0.007$  &  & $99.9\pmb{\,|\,}99.9$   & $<10^{-3}\pmb{\,|\,}0.003$   \\
BERT(max)  & 1.0       & \checkmark  & $97.4\pmb{\,|\,}97.2$   & $<10^{-3}\pmb{\,|\,}0.0008$ &  & $99.8\pmb{\,|\,}99.9$   & $<10^{-4}\pmb{\,|\,}0.0005$  \\ 
\midrule
BERT(HgFc) & 0.0       & X & $95.0\pmb{\,|\,}95.2$   & -                           &  & $72.8\pmb{\,|\,}81.2$   & -                            \\
BERT(mean) & 0.0       & \checkmark  & $97.2\pmb{\,|\,}97.1$   & $13.9\pmb{\,|\,}23.16$      &  & $100\pmb{\,|\,}99.9$    & $80.8\pmb{\,|\,}57.6$        \\
BERT(mean) & 0.1       & \checkmark  & $97.2\pmb{\,|\,}96.8$   & $0.001\pmb{\,|\,}0.006$     &  & $99.9\pmb{\,|\,}99.9$   & $<10^{-3}\pmb{\,|\,}0.001$   \\
BERT(mean) & 1.0       & \checkmark  & $97.2\pmb{\,|\,}97.1$   & $<10^{-3}\pmb{\,|\,}0.002$  &  & $99.9\pmb{\,|\,}99.8$   & $<10^{-3}\pmb{\,|\,}0.001$   \\ 
\midrule
BERT(HgFc) & 0.0       & X & $95.0\pmb{\,|\,}95.2$   & -                           &  & $72.8\pmb{\,|\,}81.24$  & -                            \\
BERT(max)  & 0.0       & \checkmark  & $97.2\pmb{\,|\,}97.2$   & $99.7\pmb{\,|\,}66,88$      &  & $100\pmb{\,|\,}99.9$    & $99.7\pmb{\,|\,}93.7$        \\
BERT(max)  & 0.1       & \checkmark  & $97.1\pmb{\,|\,}97.0$   & $<10^{-3}\pmb{\,|\,}0.003$  &  & $99.9\pmb{\,|\,}99.8$   & $<10^{-3}\pmb{\,|\,}0.006$   \\
BERT(max)  & 1.0       & \checkmark  & $97.4\pmb{\,|\,}97.0$   & $<10^{-3}\pmb{\,|\,}0.001$  &  & $99.8\pmb{\,|\,}99.9$   & $<10^{-4}\pmb{\,|\,}0.001$   \\
\bottomrule
\end{tabular}
\end{table}

\begin{table}
\centering
\caption{Classification results from Table 3 in \citet{pruthi-etal-2020-learning} for dataset SST + Wiki with cell scheme \textit{author | reproduced} for all models except BERT(HgFc) which follows cell scheme \textit{author | replicated}. Our values are means over 5 different seeds.}
\label{tab:classification-results-2}
\begin{tabular}{lcccc} 
\toprule
Model      & $\lambda$ & I & \multicolumn{2}{c}{SST + Wiki}                         \\ 
\cline{4-5}
           &           &   & Accuracy                    & Attention Mass                       \\ 
\midrule
Embedding  & 0.0       & X & $48.9 \pmb{\,|\,} 49.3$ & -                            \\
Embedding  & 0.0       & \checkmark  & $70.7 \pmb{\,|\,} 68.1$ & $48.4 \pmb{\,|\,} 49.9$      \\
Embedding  & 0.1       & \checkmark  & $67.9\pmb{\,|\,}69.5$   & $36.4\pmb{\,|\,}16.9$        \\
Embedding  & 1.0       &  \checkmark & $48.4\pmb{\,|\,}51.8$   & $8.70\pmb{\,|\,}12.9$        \\ 
\midrule
BiLSTM     & 0.0       & X & $49.1\pmb{\,|\,}48.9$   & -                            \\
BiLSTM     & 0.0       & \checkmark  & $76.9\pmb{\,|\,}75.9$   & $77.7\pmb{\,|\,}81.5$        \\
BiLSTM     & 0.1       & \checkmark  & $60.6\pmb{\,|\,}65.1$   & $0.04\pmb{\,|\,}0.99$        \\
BiLSTM     & 1.0       & \checkmark  & $61.0\pmb{\,|\,}64.9$   & $0.07\pmb{\,|\,}0.035$       \\ 
\midrule
BERT       & 0.0       & X & $50.4\pmb{\,|\,}50.2$   & -                            \\
BERT(mean) & 0.0       &  \checkmark & $90.8\pmb{\,|\,}91.8$   & $59.0\pmb{\,|\,}17.4$        \\
BERT(mean) & 0.1       &  \checkmark & $90.9\pmb{\,|\,}91.5$   & $<10^{-2}\pmb{\,|\,}0.04$    \\
BERT(mean) & 1.0       & \checkmark  & $90.6\pmb{\,|\,}91.9$   & $<10^{-3}\pmb{\,|\,}0.005$   \\ 
\midrule
BERT       & 0.0       & X & $50.4\pmb{\,|\,}50.2$   & -                            \\
BERT(max)  & 0.0       & \checkmark  & $90.8\pmb{\,|\,}91.8$   & $96.2\pmb{\,|\,}67.4$        \\
BERT(max)  & 0.1       &  \checkmark & $90.7\pmb{\,|\,}91.9$   & $<10^{-2}\pmb{\,|\,}0.04$    \\
BERT(max)  & 1.0       &  \checkmark & $90.2\pmb{\,|\,}91.8$   & $<10^{-3}\pmb{\,|\,}0.003$   \\ 
\midrule
BERT(HgFc) & 0.0       & X & $50.4\pmb{\,|\,}52.8$   & -                            \\
BERT(mean) & 0.0       & \checkmark  & $90.8\pmb{\,|\,}91.2$   & $59.0\pmb{\,|\,}68.66$       \\
BERT(mean) & 0.1       & \checkmark  & $90.9\pmb{\,|\,}90.7$   & $<10^{-2}\pmb{\,|\,}0.018$   \\
BERT(mean) & 1.0       & \checkmark  & $90.6\pmb{\,|\,}85.4$   & $<10^{-3}\pmb{\,|\,}0.019$   \\ 
\midrule
BERT(HgFc) & 0.0       & X & $50.4\pmb{\,|\,}52.8$   & -                            \\
BERT(max)  & 0.0       & \checkmark  & $90.8\pmb{\,|\,}91.1$   & $96.2\pmb{\,|\,}93.94$       \\
BERT(max)  & 0.1       & \checkmark  & $90.7\pmb{\,|\,}89.8$   & $<10^{-2}\pmb{\,|\,}0.007$   \\
BERT(max)  & 1.0       & \checkmark  & $90.2\pmb{\,|\,}86.0$   & $<10^{-3}\pmb{\,|\,}0.0007$  \\
\bottomrule
\end{tabular}
\end{table}

\begin{table}
\centering
\caption{Results for Seq2seq synthetic data tasks Bigram Flip and Sequence Copy from Table 4 in \citet{pruthi-etal-2020-learning} with cell scheme \textit{author | reproduced}. All values are means over 5 different seeds. Standard deviations are presented in Tables \ref{tab:seq-to-seq-stdv-1} and \ref{tab:seq-to-seq-stdv-2} of the Appendix.}
\label{tab:authors-seq-to-seq-results-1}
\begin{tabular}{lcccccc} 
\toprule
Attention   & $\lambda$ & \multicolumn{2}{c}{Bigram Flip}               &  & \multicolumn{2}{c}{Sequence Copy}               \\ 
\cline{3-4}\cline{6-7}
            &           & Accuracy                  & Attention Mass                  &  & Accuracy                  & Attention Mass                    \\ 
\midrule
Dot-Product & 0.0       & $100\pmb{\,|\,}100$   & $94.5\pmb{\,|\,}93.9$ &  & $99.9\pmb{\,|\,}100$  & $98.8\pmb{\,|\,}94.1$   \\ 
\midrule
Uniform     & 0.0       & $97.8\pmb{\,|\,}95.1$ & $5.2\pmb{\,|\,}4.71$  &  & $93.8\pmb{\,|\,}79.3$ & $5.2\pmb{\,|\,}4.73$    \\
None        & 0.0       & $96.4\pmb{\,|\,}96.4$ & -                     &  & $84.1\pmb{\,|\,}87.3$ & -                       \\ 
\midrule
Manipulated & 0.1       & $99.9\pmb{\,|\,}100$  & $24.4\pmb{\,|\,}15.2$ &  & $100.0\pmb{\,|\,}100$ & $27.3\pmb{\,|\,}10.7$   \\
Manipulated & 1.0       & $99.8\pmb{\,|\,}99.6$ & $0.03\pmb{\,|\,}0.01$ &  & $92.9\pmb{\,|\,}99.9$ & $0.02\pmb{\,|\,}0.014$  \\
\bottomrule
\end{tabular}
\end{table}

\begin{table}
\centering
\caption{Results for Seq2seq synthetic task Sequence Reverse from Table 4 in \citet{pruthi-etal-2020-learning} with cell scheme \textit{author | reproduced}. All values are means over 5 different seeds. Standard deviations are presented in Tables \ref{tab:seq-to-seq-stdv-1} and \ref{tab:seq-to-seq-stdv-2} of the Appendix.}
\label{tab:authors-seq-to-seq-results-2}
\begin{tabular}{lccc} 
\toprule
Attention   & $\lambda$ & \multicolumn{2}{c}{Sequence Reverse}            \\ 
\cline{3-4}
            &           & Accuracy                  & Attention Mass                  \\ 
\midrule
Dot-Product & 0.0       & $100.0\pmb{\,|\,}100$ & $94.1\pmb{\,|\,}94.0$   \\ 
\midrule
Uniform     & 0.0       & $88.1\pmb{\,|\,}80.8$ & $4.7\pmb{\,|\,}7.74$    \\
None        & 0.0       & $84.1\pmb{\,|\,}87.2$ & -                       \\ 
\midrule
Manipulated & 0.1       & $100\pmb{\,|\,}100$   & $27.6\pmb{\,|\,}16.3$   \\
Manipulated & 1.0       & $99.8\pmb{\,|\,}99.9$ & $0.01\pmb{\,|\,}0.014$  \\
\bottomrule
\end{tabular}
\end{table}

\begin{table}
\centering
\caption{Reproductions (means over 5 different seeds) En-De MT tasks from Table 4 in \citet{pruthi-etal-2020-learning} with cell scheme \textit{author | reproduced}. The BLEU(NLTK) values are not contained in the original paper, thus replicated. Standard deviations are presented in Table \ref{tab:seq-to-seq translation results stdv} of the Appendix.}
\label{tab:seq-to-seq translation results}
\begin{tabular}{lccccc} 
\toprule
Attention   & $\lambda$ & BLEU (C-MT)             & BLEU (NLTK) & Accuracy                & Attention Mass           \\ 
\midrule
Dot-Product & 0.0       & $24.42\pmb{\,|\,}24.89$ & $24.89$     & $36.99\pmb{\,|\,}36.75$ & $20.66\pmb{\,|\,}24.52$  \\ 
\midrule
Uniform     & 0.0       & $18.49\pmb{\,|\,}18.37$ & $18.37$     & $32.31\pmb{\,|\,}31.76$ & $5.96\pmb{\,|\,}5.96$    \\
None        & 0.0       & $14.89\pmb{\,|\,}15.88$ & $15.88$     & $29.73\pmb{\,|\,}30.36$ & -                        \\ 
\midrule
Manipulated & 0.1       & $23.69\pmb{\,|\,}24.30$ & $24.30$     & $36.28\pmb{\,|\,}36.49$ & $7.02\pmb{\,|\,}16.77$   \\
Manipulated & 1.0       & $20.66\pmb{\,|\,}21.07$ & $21.07$     & $33.82\pmb{\,|\,}33.68$ & $1.16\pmb{\,|\,}1.40$    \\
\bottomrule
\end{tabular}
\end{table}

% \subsection{Results Reproducing Original Paper}
% For each experiment, say 1) which claim in Section~\ref{sec:claims} it supports, and 2) if it successfully reproduced the associated experiment in the original paper.
% For example, an experiment training and evaluating a model on a dataset may support a claim that that model outperforms some baseline.
% Logically group related results into sections.

\subsection{Classification}
For the classification results, see Tables \ref{tab:classification-results-1} and \ref{tab:classification-results-2}. As can be observed, our results support claim 1, and we were able to reproduce the results from Table 3 in the original paper (excluding the Reference Letters dataset). Furthermore, we replicated the BERT(max) and BERT(mean) results using our implementation. While the results for the BERT replication match the authors' findings closely, there are also some noticeable differences; particularly, for the SST+WIKI task, we can see that for both the \textit{mean} and \textit{max} models for $\lambda = 1.0$, the replication of BERT does not manage to retain its performance, with a drop in accuracy of 4 and 6 percent, respectively.

\subsection{Seq2seq}
The results in Tables \ref{tab:authors-seq-to-seq-results-1}, \ref{tab:authors-seq-to-seq-results-2} and \ref{tab:seq-to-seq translation results} support claims 2-3, and we were able to reproduce the results from Table 4 in the original paper. Especially the reported mean accuracy by \cite{pruthi-etal-2020-learning} shows no significant difference to our reported values for all seq2seq tasks except for the baseline experiments with uniform and no attention for the tasks sequence copy and sequence reverse. Both have a mean difference of 3-14 \% accuracy regarding the authors' accuracies. \citet{pruthi-etal-2020-learning} did not report accuracies for the translation task in their original paper, but they provided us with additional raw data which also contained the accuracy scores from their experiments. Therefore we also compare accuracies for the En-De MT task.

\section{Results Beyond Original Paper}
% Often papers don't include enough information to fully specify their experiments, so some additional experimentation may be necessary. For example, it might be the case that batch size was not specified, and so different batch sizes need to be evaluated to reproduce the original results. Include the results of any additional experiments here. Note: this won't be necessary for all reproductions.

The authors state that their seq2seq results in Table 4 of the original paper are based on the average of 5 different seeds. Additional to their work we have examined the standard deviations alongside the average for all the results, comparing their results and ours. Further, while the authors do not state whether classification results from Table 3 in their original paper are based on the average of 5 seeds, we have again completed 5 runs of the experiments and provided the average.

\subsection{Under-Parameterised Models}
In the original paper, the classification results for the Embedding and BiLSTM models for the task on SST+Wiki are outliers because, while the attention mass over the impermissible tokens decreases as $\lambda$ increases, the test accuracy also decreases significantly. The authors speculate that this behavior is due to the models being under-parameterised. We investigated this by training Embedding and BiLSTM models with larger embedding dimensions. In particular, we compared embedding sizes of 128 (original size), 256, and 512. The results are presented in Tables \ref{tab:embedding-results-1} and \ref{tab:embedding-results-2}. Increasing the dimensionality of the embedding does not seem to prevent the accuracy from dropping for larger values of $\lambda$. We speculate that the drop in accuracy is due to the way the impermissible tokens are defined for the SST+Wiki dataset. All the words belonging to the SST sentence are labeled as impermissible and will therefore be penalised by the auxiliary loss component. Because the Wikipedia sentence does not provide useful information for the sentiment prediction, the model cannot rely on it and the accuracy reduces as the penalty term increases. This behaviour does not occur for the other datasets because only a few impermissible tokens were selected for the experiments, allowing the model to find other proxy tokens carrying information about the respective classification tasks. For example, words such as "lesbian" or gendered names such as "Mark" were not labeled as impermissible in the Occupation Prediction dataset. An alternative experiment could have been to define the impermissible tokens of the SST+Wiki dataset as the words in the SST sentence with the strongest positive or negative sentiment scores.

% \usepackage{booktabs}


\begin{table}[b]
\centering
\caption{Influence of the embedding size for Embedding and BiLSTM models on the SST + Wiki dataset. The values reported are the means over 5 different seeds and the standard deviations.}
\label{tab:embedding-results-1}
\begin{tabular}{lcccllll} 
\toprule
          &           &   & \multicolumn{5}{c}{Embedding dimension}                                                             \\
Model     & $\lambda$ & I & \multicolumn{2}{c}{128 (original size)}   &  & \multicolumn{2}{c}{256}                              \\ 
\cline{4-5}\cline{7-8}
          &           &   & Accuracy           & \multicolumn{1}{c}{Attention Mass} &  & \multicolumn{1}{c}{Accuracy} & \multicolumn{1}{c}{Attention Mass}  \\ 
\midrule
Embedding & 0.0       &   & $68.1 \pm 1.6$ & $49.9 \pm 2.2$           &  & $69.1 \pm 2$             & $50 \pm 1.1$              \\
Embedding & 0.1       &   & $69.5 \pm 1.4$ & $17 \pm 1$               &  & $69.1 \pm 1.6$           & $40.3 \pm 0.48$           \\
Embedding & 1.0       &   & $51.8 \pm 1.1$ & $12.9 \pm 2$             &  & $51 \pm 0.84$            & $12.3 \pm 1.4$            \\ 
\midrule
BiLSTM    & 0.0       &   & $76.4 \pm 0.8$ & $81.5 \pm 7.5$           &  & $76 \pm 2.7$             & $85.5 \pm 3$              \\
BiLSTM    & 0.1       &   & $65.1 \pm 4$   & $0.99 \pm 1.3$           &  & $65.4 \pm 4.7$           & $0.94 \pm 0.8$            \\
BiLSTM    & 1.0       &   & $64.9 \pm 3.1$ & $0.035 \pm 0.02$         &  & $62.1 \pm 2.5$           & $0.11 \pm 0.09$           \\
\bottomrule
\end{tabular}
\end{table}

\begin{table}[H]
\centering
\caption{Influence of the embedding size for Embedding and BiLSTM models on the SST + Wiki dataset. The values reported are the means over 5 different seeds and the standard deviations.}
\label{tab:embedding-results-2}
\begin{tabular}{lccll} 
\toprule
          &           &   & \multicolumn{2}{c}{Embedding dimension}              \\
Model     & $\lambda$ & I & \multicolumn{2}{c}{512}                              \\ 
\cline{4-5}
          &           &   & \multicolumn{1}{c}{Accuracy} & \multicolumn{1}{c}{Attention Mass}  \\ 
\midrule
Embedding & 0.0       &   & $68.1 \pm 2.4$           & $51.8 \pm 3.2$            \\
Embedding & 0.1       &   & $68.3 \pm 0.76$          & $41.4 \pm 2.3$            \\
Embedding & 1.0       &   & $51.8 \pm 2$             & $11.1 \pm 2.2$            \\ 
\midrule
BiLSTM    & 0.0       &   & $76.2 \pm 2.8$           & $76.8 \pm 2.4$            \\
BiLSTM    & 0.1       &   & $62.3 \pm 4.1$           & $3.5 \pm 1.9$             \\
BiLSTM    & 1.0       &   & $57.3 \pm 1.5$           & $0.13 \pm 0.9$            \\
\bottomrule
\end{tabular}
\end{table}

\section{Discussion}
% Give your judgement on if your experimental results support the claims of the paper. Discuss the strengths and weaknesses of your approach - perhaps you didn't have time to run all the experiments, or perhaps you did additional experiments that further strengthened the claims in the paper.

Our results reproduce \citet{pruthi-etal-2020-learning}'s finding that models can learn to deceive.
\cite{jain2019attention} note that for attention to be an explanation, a different configuration of attention weights for the same piece of text should lead to different predictions. The research which we have reproduced implies that the same accuracy (hence prediction) can be maintained while explicitly changing the configuration of attention weights. The implications are clear; either it is providing further evidence for why attention should not be thought of as an explanation, supporting \citet{serrano2019attention}'s findings that attention weights can be largely zeroed out without affecting accuracy. Or, if attention is an explanation, then models can be still be trained to change the attention-based explanation given and deceive algorithmic audit. This research thus provides a new vein of the investigation into the attention-based explanation debate.

Examination of the standard deviations showed whether reproduction differences were meaningful for seq2seq tasks. Regarding synthetic data, it showed some variance from the authors' values for attention mass, but it was more strongly in the experimental direction, thus supporting their findings. For machine translation, one result at $\lambda = 0.1$ for attention mass was not as strong as their result, but still trending in the experimental direction. While the standard deviation does show that there is some variance inherent in the reduced attention masses under manipulation, it provides still further robustness to the findings.

Finally, we obtain very similar results with our replication of the authors' BERT-based classification model, for both the \textit{mean} and \textit{max} penalties, for the majority of the tasks. The only results which we did not manage to exactly replicate concern the test accuracies for \textit{SST+Wiki}. However, given that we did manage to \textit{reproduce} these results with the authors' implementation of BERT, it is more likely that the differences in test accuracies between the reproduction and replication experiments can be attributed to slight differences in hyperparameter settings between the two - however, further investigation would be required to confirm this. Further to this, we observe somewhat large differences between the attention masses - particularly for values where $\lambda = 0$. However, given that similar differences persist also for our \textit{reproduction} results of BERT, it may be the case that these quantities are simply more susceptible to stochasticity and/or training dynamics, which would then explain the observed discrepancies with the authors' findings. Either way, these particular differences do not provide grounds to reject the authors' hypotheses; we still observe that for most replication runs, the accuracy is not impacted despite substantial reductions in attention masses on the impermissible tokens. To this end, our replication efforts provide further evidence for the authors' claims that attention-as-explanation can be deceptive.

\subsection{What Was Easy}
% Give your judgement of what was easy to reproduce. Perhaps the authors' code is clearly written and easy to run, so it was easy to verify the majority of original claims. Or, the explanation in the paper was really easy to follow and put into code.
% Be careful not to give sweeping generalizations. Something that is easy for you might be difficult to others. Put what was easy in context and explain why it was easy (e.g. code had extensive API documentation and a lot of examples that matched experiments in papers).

Overall, we could easily understand and follow the authors' descriptions of their methods in the paper. An example of this is the way seq2seq tasks and the datasets used were described in a short but comprehensive way. The provided bash scripts revealed the basic setup and could almost immediately be used to run the experiments on our infrastructure described in Section \ref{sec:infrastructure}. Therefore, we were quickly able to reproduce the first results. The codebase for seq2seq tasks was easily restructured into functions, instead of keeping the train and evaluation functionality at a global code level. Besides this, applying PEP8 to the codebase was an easy task - another positive is that restructuring did not break the code massively at any point, which in our opinion testifies to an already consistent architecture.
% Both served the purpose to use the code as a black box, in our case from a Jupyter notebook for displaying the results, without the exact knowledge how the code is internally structured that is for example without knowing which global variables would have to be declared for the experiment to run properly.
The authors' integration of the BLUE score implementation (compare-mt) was missing. However, we could easily add the NLTK BLEU score implementation into the code. We could further observe robust BLEU score results, like the ones we reproduced and replicated turned out to be not significantly different from those reported by the authors.

\subsection{What Was Difficult}
% List part of the reproduction study that took more time than you anticipated or you felt were difficult.
% Be careful to put your discussion in context. For example, don't say "the maths was difficult to follow", say "the math requires advanced knowledge of calculus to follow".

% BERT implementation
To a certain extent, it was achievable to re-implement the BERT-based model using only the information provided in the original work. Nevertheless, in the process of re-implementing, several ambiguities arose that we were initially unable to resolve and which had to be clarified by the authors. For instance, the penalty mechanism used to compute $\mathcal{R}$ by default assumed an attention \textit{vector} $\boldsymbol{\alpha}$, while BERT, by default, outputs self-attention \textit{matrices}. We were initially unsure how to obtain a vector from this matrix; only after contacting the authors, we learned that $\boldsymbol{\alpha}$ can be obtained from BERT's self-attention matrices by only considering the first row of the matrix (for a given self-attention head), which represents the extent to which the \texttt{[CLS]} token attends to all other tokens in the sentence. The authors also further clarified that they only used the attention output of the last (12th) transformer block of the model, whereas we initially understood from the paper that all layers should be taken into account.

Additionally, there was some brief confusion relating to the evaluation procedure. In their papers' Section 5.1, the authors provide a heuristic to ``pick'' the best deceptive model. Based on how this procedure was formulated, we initially believed that multiple models were trained for a given epoch sequentially, after which the best model (as evaluated by admissible accuracy and largest reduction in attention mass) was selected for the next epoch. However, after corresponding with the authors, we learned that rather, a single model is simply trained for 10 epochs, and only then the selection heuristic is applied manually to determine which model \textit{checkpoint} will be used for evaluation on the test set. In these respects, replicating the model architecture, penalty mechanism, and training procedure might have been easier, had the original work been more precise and explicit regarding its methods.

We also chose to port the code to PyTorch Lightning to make it easier to reproduce the research in the future, but this necessitated changes to data loading, pre-processing, and batching. A specific challenge was PyTorch lightning not yet supporting checkpointing over multiple metrics out-of-the-box, meaning we had to implement the authors' multi-metric heuristic ourselves.
% Next to this, there was a small complication in our attempts to port everything to PyTorch Lightning; for selecting the best adversarial model (models with $\lambda = {0.1, 1.0}$), the authors use a multi-metric heuristic, i.e. a checkpoint should 1) have a validation accuracy which is at least within a -2\% range of the $\lambda = 0$ baseline model test accuracy, whilst also offering the largest reduction in attention mass, compared to the aforementioned baseline. While we believe that the authors did this by hand (to our knowledge, the original repository does not feature code for this heuristic), we felt that it would be both more robust and convenient to automate this process. However, since Lightning does not yet support checkpointing over multiple metrics out-of-the-box, we had to implement this functionality ourselves.
% Further, due to BERT-specific pipelines (e.g. word piece tokenisation) implementing the transformer model in PyTorch Lighting, with its modularised coding style, was challenging.
% Finally, it should be noted that by design, a BERT-based framework typically requires a somewhat unique pipeline (for instance, it requires the usage of a BERT-specific tokeniser, due to BERTs reliance on word piece tokenisation). Because of this, the modularised coding style required by PyTorch Lightning, and due to the initial unavailability of the authors' BERT implementation, the BERT-based re-implementation was done entirely from scratch, and is therefore entirely disjoint from both the authors' codebase, and our own \textit{reproduction} code.

\subsection{Communication With Original Authors}
% Document the extent of (or lack of) communication with the original authors. To make sure the reproducibility report is a fair assessment of the original research we recommend getting in touch with the original authors. You can ask authors specific questions, or if you don't have any questions you can send them the full report to get their feedback before it gets published.

Following our initial contact email, the authors made themselves readily available, quickly responding to a series of emails over multiple weeks, answering all questions clearly, and providing access to everything we required to reproduce their results. We have also provided the authors with this full report.

\newpage