% DO NOT EDIT - automatically generated from metadata.yaml

\def \codeURL{https://github.com/tzole1155/EndToEndObjectPose}
\def \codeDOI{}
\def \codeSWH{swh:1:dir:5e754c06b0dac970857b31733766fb9a268edee6}
\def \dataURL{https://vcl3d.github.io/UAVA/}
\def \dataDOI{10.5281/zenodo.3994337}
\def \editorNAME{}
\def \editorORCID{}
\def \reviewerINAME{}
\def \reviewerIORCID{}
\def \reviewerIINAME{}
\def \reviewerIIORCID{}
\def \dateRECEIVED{}
\def \dateACCEPTED{}
\def \datePUBLISHED{}
\def \articleTITLE{On end-to-end 6{DoF} object pose estimation and robustness to object scale}
\def \articleTYPE{Replication}
\def \articleDOMAIN{ML Reproducibility Challenge 2020}
\def \articleBIBLIOGRAPHY{bibliography.bib}
\def \articleYEAR{2021}
\def \reviewURL{https://openreview.net/forum?id=PCpGvUrwfQB}
\def \articleABSTRACT{ This report contains a set of experiments that seek to reproduce the claims of two recent works related to keypoint estimation, one specific to 6DoF object pose estimation, and the other presenting a generic architectural improvement for keypoint estimation but demonstrated in human pose estimation. More specifically, in the backpropagatable PnP , the authors claim that incorporating geometric optimization in a deep-learning pipeline and predicting an object’s pose in an end-to-end manner yields improved performance. On the other hand, HigherHRNet  introduces a novel heatmap aggregation method that allows for scale-aware pose estimations, offering higher keypoint localization accuracy for small scale objects. }
\def \replicationCITE{chen2020end, title={End-to-end learnable geometric vision by backpropagating PnP optimization}, author={Chen, Bo and Parra, Alvaro and Cao, Jiewei and Li, Nan and Chin, Tat-Jun}, booktitle={Proceedings of the IEEE/CVF Conference on Computer Vision and Pattern Recognition}, pages={8100--8109}, year={2020}}
\def \replicationBIB{chen2020end}
\def \replicationURL{https://openaccess.thecvf.com/content_CVPR_2020/papers/Chen_End-to-End_Learnable_Geometric_Vision_by_Backpropagating_PnP_Optimization_CVPR_2020_paper.pdf}
\def \replicationDOI{}
\def \contactNAME{Georgios Albanis}
\def \contactEMAIL{galbanis@iti.gr}
\def \articleKEYWORDS{rescience c, rescience x , python, pytorch, object pose estimation}
\def \journalNAME{None}
\def \journalVOLUME{}
\def \journalISSUE{}
\def \articleNUMBER{}
\def \articleDOI{}
\def \authorsFULL{Georgios Albanis et al.}
\def \authorsABBRV{G. Albanis et al.}
\def \authorsSHORT{Albanis et al.}
\title{\articleTITLE}
\date{}
\author[1,\orcid{0000-0002-2032-6767}]{Georgios Albanis}
\author[1,\orcid{0000-0002-7898-9344}]{Nikolaos Zioulis}
\author[1,\orcid{0000-0002-3848-4210}]{Anargyros Chatzitofis}
\author[1,\orcid{0000-0003-2763-4217}]{Anastasios Dimou}
\author[1,\orcid{0000-0002-9649-9306}]{Dimitrios Zarpalas}
\author[1,\orcid{0000-0003-3814-6710}]{Petros Daras}
\affil[1]{Centre for Research and Technology Hellas, Thessaloniki, Greece}
