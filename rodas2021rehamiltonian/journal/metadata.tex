% DO NOT EDIT - automatically generated from metadata.yaml

\def \codeURL{https://github.com/CampusAI/Hamiltonian-Generative-Networks}
\def \codeDOI{10.5281/zenodo.4673473}
\def \codeSWH{swh:1:dir:7c7cee4b1c4dc7dfa46b4b4bf5c3b8b20deb5b26}
\def \dataURL{}
\def \dataDOI{}
\def \editorNAME{Koustuv Sinha,\\ Sasha Luccioni}
\def \editorORCID{}
\def \reviewerINAME{Anonymous Reviewers}
\def \reviewerIORCID{}
\def \reviewerIINAME{}
\def \reviewerIIORCID{}
\def \dateRECEIVED{29 January 2021}
\def \dateACCEPTED{01 April 2021}
\def \datePUBLISHED{27 May 2021}
\def \articleTITLE{[Re] Hamiltonian Generative Networks}
\def \articleTYPE{Replication}
\def \articleDOMAIN{ML Reproducibility Challenge 2020}
\def \articleBIBLIOGRAPHY{../openreview/bibliography.bib}
\def \articleYEAR{2021}
\def \reviewURL{https://openreview.net/forum?id=Zszk4rXgesL}
\def \articleABSTRACT{Hamilton's equations are widely used in classical and quantum physics. The Hamiltonian Generative Network (HGN) is the first approach that aims to "learn the Hamiltonian dynamics of simple physical systems from high-dimensional observations without restrictive domain assumptions". To do so, a variational model is trained to reconstruct the evolution of physical systems directly from images by integrating the learned Hamiltonian. New trajectories can be sampled and rollouts can be performed forward and backward in time. In this work, we re-implement the HGN architecture and the physical environments (pendulum, body-spring system, and 2,3-bodies). We reproduce the paper experiments and we further expand them by testing on two new environments and one new integrator. Overall, we find that obtaining both good reconstruction and generative capabilities is hard and sensitive to the variational parameters.}
\def \replicationCITE{Peter Toth, Danilo Jimenez Rezende, Andrew Jaegle, Sébastien Racanière, Aleksandar Botev, and Irina Higgins. Hamiltonian generative networks, 2020}
\def \replicationBIB{toth2020hamiltonian}
\def \replicationURL{https://arxiv.org/abs/1909.13789}
\def \replicationDOI{}
\def \contactNAME{Taschin, Federico}
\def \contactEMAIL{taschin@kth.se}
\def \articleKEYWORDS{Hamiltonian, generative network, variational, python, pytorch}
\def \journalNAME{ReScience C}
\def \journalVOLUME{7}
\def \journalISSUE{2}
\def \articleNUMBER{18}
\def \articleDOI{10.5281/zenodo.4835278}
\def \authorsFULL{Balsells Rodas, Carles, Canal Anton, Oleguer and Taschin, Federico}
\def \authorsABBRV{C. Balsells Rodas, O. Canal Anton and F. Taschin}
\def \authorsSHORT{Balsells Rodas, Canal Anton and Taschin}
\title{\articleTITLE}
\date{}
\author[1,\orcid{0000-0001-5373-8920}]{Balsells Rodas, Carles}
\author[1,\orcid{0000-0001-6436-3829}]{Canal Anton, Oleguer}
\author[1,\orcid{0000-0001-6240-2521}]{Taschin, Federico}
\affil[1]{KTH Royal Institute of Technology, CNRS UPR4301, Stockholm, Sweden}
