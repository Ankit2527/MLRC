\section{Introduction}

% A  few  sentences  placing  the  work  in  context. Limit it to a few paragraphs at most; your report is on reproducing a piece of work, you don’t have to motivate that work.

% Problem
Consider an isolated physical system with multiple bodies interacting with each other.
Let $\bm{q} \in \mathbb{R}^n$ be the vector of their positions, and $\bm{p} \in \mathbb{R}^n$ the vector of their momenta.
The Hamiltonian formalism \cite{hamilton} states that there exists a function $\mathcal{H} : (\bm{q}, \bm{p}) \in \mathbb{R}^{n + n} \rightarrow \mathbb{R}$ representing the energy of the system which relates $\bm{q}$ and $\bm{p}$ as:
\begin{equation}
\frac{\partial \bm{q}}{\partial t} = \frac{\partial \mathcal{H}}{\partial \bm{p}}, \qquad
\frac{\partial \bm{p}}{\partial t} = -\frac{\partial \mathcal{H}}{\partial \bm{q}}
\label{eq:hamilton}
\end{equation}
% Approach
In this work $\mathcal{H}$ is modeled with an artificial neural network and property \ref{eq:hamilton} is exploited to get the temporal derivatives of both $\bm{q}$ and $\bm{p}$.
One can then use a numerical integrator (see Section \ref{sec:integrators}) to solve the ODE and infer the system evolution both forward and backward in time given some initial conditions (see Figure \ref{fig:unroll}).
These initial conditions are inferred from a natural image sequence of the system evolution (see Figure \ref{fig:initial_cond}).
The authors propose a generative approach to learn low-dimensional representations of the positions and momenta $(\bm{q}_0, \bm{p}_0)$.
This allows us to sample new initial conditions and unroll previously unseen system evolutions according to the learned Hamiltonian dynamics.

